
 
\documentstyle[11pt]{article}
%\input{epsf}
 
\setlength{\oddsidemargin}{0.260in}
\setlength{\evensidemargin}{0.260in}
\setlength{\topmargin}{-0.20in}
 
 
%\setlength{\headsep}{0mm}
%\setlength{\headheight}{0mm}
%\setlength{\textwidth}{7in}
%\setlength{\textheight}{9.375in}
%\setlength{\columnsep}{0.33in}
%\setlength{\baselineskip}{0.5\normalbaselineskip}
%\setlength{\parindent}{8mm}
%\renewcommand{\topmargin}{0in}
\setlength{\textheight}{9.0in}
\setlength{\textwidth}{6.5in}
%\renewcommand{\footskip}{2in}
\renewcommand{\baselinestretch}{1.00}
\newtheorem{theorem}{Theorem}[section]
\newtheorem{algorithm}{Algorithm}[section]
\newtheorem{lemma}{Lemma}[section]
\newtheorem{observation}{Observation}[section]
\newtheorem{property}{Property}[section]
\newtheorem{problem}{Problem}[section]
\newtheorem{definition}{Definition}[section]
\newtheorem{remark}{Remark}[section]
 
\begin{document}
 
\pagestyle{empty}


\begin{center}
             {\Large Cost Constrainted Shortest
		Paths }\\
		CS 351\\
\end{center}

\section{Description}
Imagine the following situation.  You are trying to travel
across the country (from start city $s$ to destination city $d$), 
but have a limited budget.  To
travel between two cities there may be many paths and different
choices will have different tradeoffs between cost and time --
e.g., taking a bus may be cheap but take a long time; 
renting a car may be fast, but expensive.

Essentially, what you have then is a doubly weighted graph.
One weight is the {\em cost} $c$ and the other is the {\em traversal time}
$t$.  We will assume that 
all {\em cost} and {\em traversal time} values are 
{\bf non-negative integers}.

Your goal then is to determine the fastest way to get to your destination
within your budget.


\section{Algorithm Sketch}
This section gives an overview of the main ideas behind an
algorithm solving this problem.  Part of your job will be to foramlize
it into an actual implemtation.

The algorithm has a Dijkstra-like sturture with some important
differences.

\begin{itemize}
\item In Dijkstra, we store a label \verb+d[v]+ at each vertex.  For
	our problem we will have to store a {\em list} of candidate
	paths to $v$ and the ``signature'' for each such path (i.e.,
	the total cost of the path and the total traversal time
	of the path).  
\item For a vertex $v$, we only need to store ``non-dominated'' path
	signatures.  A path is {\em dominated} if some other path 
	is better in one of the dimensions and better (less than) than or
	equal in the other (i.e., there is no point in using
	the first path).  We will call this set \verb+S[v]+
\item In a manner similar to Dijkstra, we will expand in non-decreasing
	order of {\em path cost}.
\item Because of this order of expansion and the fact that
	\verb+S[v]+ contains only non-dominated signatures, we
	can order the set in {\em strictly increasing order} of
	cost and {\em strictly decreasing order of traversal time}.
\item An invariant will be that \verb+S[v]+ will never have anything
	deleted from it.  In other words, no path to $v$ discovered
	in the future will make any of the signatures in \verb+S[v]+
	dominated.  The next bullet will clarify this.
\item In addition to maintaining \verb+S[v]+,
	we also maintain a binary heap which keeps track of the
	``frontier''.  Details follow.
 	\begin{itemize}
	\item The heap contains data elements of the 
		form $<(c,t),v>$ which indidates that
		we have discovered a path from the
		source to $v$ which has cost $c$ and traversal
		time $t$.
	\item Some of these path signatures sitting in
		the heap may turn out to be sub-optimal
		(dominated by some other path).
	\item The heap (a min-heap) is ordered first by cost and second by
		time.  This preserves the property that path signatures
		are added to the \verb+S[]+ sets in non-decreasing
		order of $c$.  By secondarily, we mean that 
		time is used as a tie-breaker in the heap (i.e.,
		it is lexicographic).
	\item Unlike Dijkstra's algorithm, we will never do a
		\verb+DecreaseKey()+ operation.  This is becaue
		for a particular vertex, we may have many candidate
		paths (signatures) sitting in the heap.  When we 
		see a new path, we just insert it.
	\item Also unlike Dijkstra's algorithm, when we remove
		a candidate signature $<(c,t),v>$ from the heap
		by a \verb+DeleteMin()+, we need to make sure
		that it is not dominated by a previously found
		solution (i.e., one already in \verb+S[v]+).
		{\bf Important:}  recall that \verb+S[v]+
		is ordered in increasing order of $c$ and
		decreasing order of $t$ and by our invariant,
		all of the signatures are non-dominated; now,
		we have just removed $<(c,t),v>$ from the heap.
		When is this solution dominated by a solution
		already in \verb+S[v]+?  We already know that
		$c$ from the newly extracted solution is at
		least as large as all $c$'s in \verb+S[v]+.
		Do any of them have smaller $t$-values?  Well,
		the last entry in \verb+S[v]+ has the smallest
		$t$-value in \verb+S[v]+, so if the new $t$-value
		from the  heap is greater than any entry in 
		\verb+S[v]+, it must be greater than the {\em last}
                $t$-value in \verb+S[v]+.  What does this mean?
                You can check if the extracted solution is dominated
                by a prior solution by one comparison with
                the last entry in \verb+S[v]+.  If it is not dominated,
		you append it; if it is dominated, you discard it.
	\item Only when you have discovered a new non-dominated
		path as above, do you ``expand'' this solution and
		add new signatures to the heap.  This is done
		in the natural way by examining edges leaving the
		vertex and ``augmenting'' the current path
		cost and time with those of the edge.  
		Before inserting this new candidate into the heap,
		you might as well do a check to make sure it
		is not already dominated (as above).  The algorithm
		still works if you don't do this check.	
	\end{itemize}
\end{itemize}

The above are the key ideas of the algorithm.  Read it multiple
times if it helps.

Recall that in the formulation, you have a budget.  The above
algorithm is the same regardless of budget, but when it is done,
you examine the non-dominated solutions at the destination and
pick the fastest one within your budget.  (You can use the budget
to make the algorithm a bit faster by stopping early when
possible, but this is not required).

\section{Pencil and Paper}
Before starting your program, you will do the following exercises/problems.

\begin{itemize}
\item[(1)] Construct an example instance of the problem with
	at least 5 vertices and at least 3 non-dominated paths
	from the specified source to the specified destination.
	Some of the paths should overlap (i.e., share common
	subpaths).
	Simulate the algorithm by hand.  You need not simulate
	the individual heap operations, just maintain the contents
	of the heap clearly.
\item[(2)] Recall that cost and time values are non-negative integers.
	\begin{itemize}
	\item Let the sum of all edge costs be $C=\sum_{e\in E)} c(e)$
	\item Let the sum of all edge times be $T=\sum_{e\in E)} t(e)$
	\item Let $M= \max(C,T)$
	\item Using $M$, give an upper bound on $|S[v]|$.  Recall that
		all cost values will be distinct (so will time values).
		Your bound will be pretty crude.
	\item Using this bound, give an upper bound on the number 
		of heap operations (insertiona and deletions).
	\item Use this result to give an upper bound on the overall
		runtime of the algorithm.
	\end{itemize}
\end{itemize}

\section{Program Specifics}
Your executable will be called \verb+cpath+ (for constrained
paths) and will have four
command line parameters.  

The usage is ``\verb+cpath <file> <s> <d> <budget>+''.
So you are given a file contains the graph, the source
vertex $s$ (an int), the destination vertex $d$ (an int)
and a budget (also an int).

You will provide a makefile which produces the executable
\verb+cpath+).

\subsection*{File Format}
The first line of the file will contain the number
of vertices.  Vertices are always implicitly numbered
$0..|V|-1$.

After this there is a sequence of edges in the form
``\verb+u v c t+'' which says there is an edge from
$u$ to $v$ with cost $c$ and traversal time $t$.
For instance ``\verb+5 2 10 5+ indicates that there
is an edge from vertex 5 to vertex 2 with cost of 10
units and traversal time of 5 units.

{\bf Assumptions:  }

\begin{itemize}
  \item You may assume that there is exactly one edge per line
to simplify input parsing.
  \item Cost and time values are given as non-negative integers.
\end{itemize}

{\bf Sample Input File:}

\begin{verbatim}

    7
    0 1 2 4
    1 3 2 1
    0 4 4 2
    4 5 3 2
    1 5 4 1
    1 2 3 2
    3 2 2 1
    2 6 2 3
    5 6 4 1

\end{verbatim}


\subsection*{Output}
After your program runs it will report the fastest cost-feasible
path (if no feasible path exists, the program reports so).
You should give the cost and traversal time of the path as
well as the path itself (as a sequence of vertices).  The format
is up to you -- just keep it simple and clear.
		
		
\section*{Things to remember}

You can/should use the STL priority queue since:
\begin{itemize}
  \item we don't need the \verb+decrese_key+ operation and
  \item we want to use ``time'' as a tie-breaker in the heap ordering
    and the heap you worked on in p2 would need to be modified to
    support this feature.
\end{itemize}







		


\end{document}


